\documentclass[10pt,a4paper]{scrartcl}
\usepackage[utf8]{inputenc}
\usepackage{booktabs}
\usepackage{graphicx}
\usepackage{caption}
\usepackage{tabularx}
\usepackage{eurosym}
\DeclareRobustCommand{\officialeuro}{%
  \ifmmode\expandafter\text\fi
  {\fontencoding{U}\fontfamily{eurosym}\selectfont e}}
\usepackage[tmargin=2.5cm,bmargin=1cm,lmargin=2.5cm,rmargin=2.0cm]{geometry}
\usepackage{helvet}
\renewcommand{\familydefault}{\sfdefault}
\newcommand{\squeezeup}{\vspace{-2.5mm}}
\newcolumntype{C}[1]{>{\centering}p{#1}}
\usepackage[overlay,absolute]{textpos}

\newcommand\PlaceText[3]{%
\begin{textblock*}{10in}(#1,#2)  %% change width of box from 10in as you wish
#3
\end{textblock*}
}%
\textblockorigin{-5mm}{0mm}   %% Default origin top left corner and it can be changed in this line
\usepackage{array}

\newlength\Origarrayrulewidth

% horizontal rule equivalent to \cline but with 2pt width
\newcommand{\Cline}[1]{%
 \noalign{\global\setlength\Origarrayrulewidth{\arrayrulewidth}}%
 \noalign{\global\setlength\arrayrulewidth{2pt}}\cline{#1}%
 \noalign{\global\setlength\arrayrulewidth{\Origarrayrulewidth}}%
}

% draw a vertical rule of width 2pt on both sides of a cell
\newcommand\Thickvrule[1]{%
  \multicolumn{1}{!{\vrule width 2pt}c!{\vrule width 2pt}}{#1}%
}

% draw a vertical rule of width 2pt on the left side of a cell
\newcommand\Thickvrulel[1]{%
  \multicolumn{1}{!{\vrule width 2pt}c|}{#1}%
}

% draw a vertical rule of width 2pt on the right side of a cell
\newcommand\Thickvruler[1]{%
  \multicolumn{1}{|c!{\vrule width 2pt}}{#1}%
}

\begin{document}
\pagenumbering{gobble}
{\large
\begin{tabular}{@{}p{3.0in}l}
\begin{tabular}{l}
An die\\
TU Braunschweig\\
Sportzentrum \\
Franz-Liszt-Straße 34\\
38106 Braunschweig\\
\end{tabular}
&
\begin{tabular}{ll}
Name &\line(1,0){180}\\
Adresse&\line(1,0){180}\\
			&\line(1,0){180}\\
IBAN&\line(1,0){180}\\
\multicolumn{2}{l}{\footnotesize\textbf{Die Angabe der IBAN ist zwingend notwendig!}}\\
\end{tabular}
\end{tabular}}
\bigskip
\bigskip
\begin{table}[ht]
    \begin{tabular}{r@{}p{3.5in}l}
      
			\includegraphics[width=3.51cm]{logo.jpg}
			&
			
			\centering\Large\textbf{TRAINERABRECHNUNG}   
			
			&\includegraphics[width=3.51cm]{logo.jpg} \\
    \end{tabular}
\end{table}\\
\bigskip
\bigskip
\normalsize
{\linethickness{0.6mm}
\textbf{Sportart: \line(1,0){170}\ \ \ \ Monat: \line(1,0){170}}}
\squeezeup\squeezeup\squeezeup
\begin{table}[ht]
	\centering
	\caption*{\large Nachstehend aufgeführte Stunden sind von mir tatsächlich geleistet worden.}
		\begin{tabularx}{\textwidth}{|C{1.5cm}|C{2,1832cm}|C{1.025cm}|C{1.025cm}|C{1.025cm}|C{1.025cm}|C{1.025cm}|C{1.025cm}|C{1.025cm}|c|}
		\hline
		\textbf{Datum}&\textbf{Arbeitszeit}&\textbf{Std.}&\textbf{6,50}&\textbf{9,00}&\textbf{11,50}&\textbf{13,50}&\textbf{15,00}&&\begin{tabular}[x]{@{}c@{}}\textbf{Teil-}\\\textbf{nehmer}\end{tabular} \\ \hline
		&&&&&&&&&\\\hline
		&&&&&&&&&\\\hline
		&&&&&&&&&\\\hline
		&&&&&&&&&\\\hline
		&&&&&&&&&\\\hline
		&&&&&&&&&\\\hline
		&&&&&&&&&\\\hline
		&&&&&&&&&\\\hline
		&&&&&&&&&\\\hline
		&&&&&&&&&\\\hline
		&&&&&&&&&\\\hline
		&&&&&&&&&\\\hline
		&&&&&&&&&\\\hline
		\multicolumn{10}{|l|}{}\\
		\multicolumn{10}{|l|}{\large Vertretungsstunden (Sportart wie oben)}\\\hline
		&&&&&&&&&\\\hline
		&&&&&&&&&\\\hline
		&&&&&&&&&\\\cline{1-2}\Cline{3-8}\cline{9-10}
		\multicolumn{1}{l}{}&\multicolumn{1}{l}{Stunden}&\Thickvrule{}&\Thickvruler{}&\Thickvruler{}&\Thickvruler{}&\Thickvruler{}&\Thickvruler{}\\
		\multicolumn{1}{l}{}&\multicolumn{1}{l}{insgesamt}&\Thickvrule{}&\Thickvruler{}&\Thickvruler{}&\Thickvruler{}&\Thickvruler{}&\Thickvruler{}\\\Cline{3-8}
		\end{tabularx}
\end{table}\\\\
\PlaceText{19.6cm}{10.7cm}{\rotatebox{90}{\footnotesize \textbf{Bitte nicht vergessen die Anzahl der Teilnehmer einzutragen!}}} 
\begin{minipage}{0.40\textwidth}
{\large Braunschweig, den \line(1,0){79}}\\\\\\\\
\line(1,0){184}\\
\squeezeup
{\large Unterschrift Trainer(in)} \\\\\\
{\large Sachlich und rechnerisch richtig:}\\\\\\\\
\line(1,0){184}\\
\squeezeup
{\large Sportzentrum }
\end{minipage}
\begin{minipage}{0.03\textwidth}
\begin{tabularx}{\textwidth}{c}
\end{tabularx}
\end{minipage}
\begin{minipage}{0.57\textwidth}
	\begin{tabularx}{\textwidth}{|l|l|l|l|l|p{2.562cm}|}
		\hline
			\multicolumn{6}{|c|}{\textbf{Eingangsrechnung}}\\
		\hline
		\footnotesize Belegart&\multicolumn{2}{|l|}{\footnotesize Buchungsdatum}&\multicolumn{3}{|l|}{\footnotesize Beleg-Nr.}\\
		KA&\multicolumn{2}{|l|}{}&\multicolumn{3}{|l|}{}\\
		\hline
		\multicolumn{2}{|l|}{\footnotesize S (Sachkonto/Kreditor)}&\footnotesize StSchl.&\multicolumn{2}{|l|}{\footnotesize Betrag in \euro{}}&\footnotesize KST/IA-Nr.\\
		\multicolumn{2}{|l|}{667 400}&&\multicolumn{2}{|l|}{}&305 000 00\\\hline
		\multicolumn{2}{|l|}{\footnotesize H (Kreditor/Sachkonto)}&\footnotesize StSchl.&\multicolumn{2}{|l|}{\footnotesize Betrag in \euro{}}&\footnotesize KST/IA-Nr.\\
		\multicolumn{2}{|l|}{}& &\multicolumn{2}{|l|}{}&305 000 00\\\hline
	\multicolumn{2}{|l|}{\footnotesize Sachlich richtig}&\multicolumn{2}{|l|}{\footnotesize Sachlich und rechnerisch }&\multicolumn{2}{|l|}{\footnotesize Rechnerisch }\\
	\multicolumn{2}{|l|}{\footnotesize Im Auftrage}&\multicolumn{2}{|l|}{\footnotesize richtig}&\multicolumn{2}{|l|}{richtig}\\
	\multicolumn{2}{|l|}{}&\multicolumn{2}{|l|}{}&\multicolumn{2}{|l|}{}\\
	\multicolumn{2}{|l|}{}&\multicolumn{2}{|l|}{}&\multicolumn{2}{|l|}{}\\
	\multicolumn{2}{|l|}{\footnotesize  (Datum, Unterschrift }&\multicolumn{2}{|l|}{}&\multicolumn{2}{|l|}{}\\
	\multicolumn{2}{|l|}{\footnotesize  Anordnungsbefugte/r)}&\multicolumn{2}{|l|}{}&\multicolumn{2}{|l|}{}\\\hline
	\multicolumn{4}{|l|}{\footnotesize Kontiert/Vorerfasst von  }&\multicolumn{2}{|l|}{\footnotesize Datum}\\
	\multicolumn{4}{|l|}{\footnotesize Gebucht/Freigegeben von   }&\multicolumn{2}{|l|}{}\\\hline
	\end{tabularx}
\end{minipage}

\end{document}
